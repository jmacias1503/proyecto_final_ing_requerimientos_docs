\chapter{Descripción General del Proyecto}
\section{Resumen}
El propósito de este proyecto es crear una herramienta de tareas, la cual pueda crear, editar, eliminar y marcar como completadas
las tareas que el usuario desee. Además, se podrá visualizar las tareas que se han completado y las que están pendientes.

\section{Alcances del proyecto}
\begin{itemize}
  \item Gestionar el proyecto mediante la herramienta JIRA utilizando la metodología Scrum.
  \item Desarrollar un prototipado con todas las funcionalidades que se dean implementar en Figma.
  \item Hacer uso un repositorio en GitHub para el control de versiones de el frontend.
  \item Hacer uso repositorio en GitHub para el control de versiones de el backend.
  \item Crear un diagrama relacional de la base de datos.
  \item Crear una base de datos con la estructura del diagrama relacional.
  \item Desarrollar la aplicación.
  \item Lanzar la aplicación en un servidor.
  \item Realizar test a la aplicación.
  \item Desglosar los pasos del proyecto en una presentación.
\end{itemize}

\section{Limitaciones del proyecto}
\begin{itemize}
  \item El desarrollo de el proyecto tiene un tiempo limitado (3 semanas y 4 días).
  \item El proyecto no busca fines lucrativos.
  \item El proyecto se encuentra limitado a la creación de tareas.
\end{itemize}


\section{Justificación}
El proyecto surge en base a un proyecto final solicitado en la clase de Ingeniería de Requerimientos
de la Universidad Autónoma de Querétaro. La idea de crear un gestor de tareas es con la finalidad de poner 
a prueba los conocimientos adquiridos en la materia y en la carrera.
